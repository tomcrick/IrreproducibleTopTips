\documentclass[a4paper,11pt]{article}
\usepackage[top=2cm,bottom=2cm,left=2cm,right=2cm,asymmetric]{geometry}
\usepackage{url}
\usepackage{paralist}
\usepackage{authblk}
\usepackage[colorlinks=true,hyperfootnotes=false]{hyperref}

\title{Top Tips for Irreproducible Research}

% add names here! alphabetical?
\author[1]{Tom Crick}
\author[2]{Neil P. Chue Hong}
\author[3]{Lars Kotthoff}
% \author[2]{Person 3...}
\affil[1]{Department of Computing \& Information Systems, Cardiff
  Metropolitan University}
  \affil[2]{Software Sustainability Institute, University of Edinburgh}
\affil[3]{Insight Centre for Data Analytics, University College Cork}
% \affil[3]{Org3}
\affil[1]{\protect\url{tcrick@cardiffmet.ac.uk}}
\affil[2]{\protect\url{N.ChueHong@software.ac.uk}}
\affil[3]{\protect\url{lars.kotthoff@insight-centre.org}}
% \affil[3]{\protect\url{email3}}

\renewcommand\Authands{ and }
\def\UrlBreaks{\do\/\do-}

\date{ }

\begin{document}
\maketitle

\begin{abstract}
Guidelines and top tips for doing irreproducible research.
\end{abstract}

% c.800 words?
\section{Introduction}
Add stuff here...

% General idea
% Come up with a paper on Irreproducibility Top Tips
% Use specific personal examples from the authors to back each tip up

% This is the list we came up with at the workshop
\begin{enumerate}
\item implement everything yourself / don't reuse existing, well tested code wherever possible
\item  don't share anything
\ don't make it clear what irreproducibility means in your context / domain
\item don't make it clear where your method / approximations cannot be / should not be used
\item don't publish your source code, or your implementation
\item  define custom benchmarks
\item  don't describe your experimental setup
\item  give a vague methodology
\item  data formats are boring
\item  use magic numbers everywhere
\item  miss out key parts / fudgy workflow
\item  don't share the raw data you used to build everything
\item  define your own data formats, preferably in binary, mont human or machine readable
\end{enumerate}

Need to distil those down to probably not more than 5 points -- here's Lars'
attempt:
\begin{enumerate}
\item Pseudo-code is a great way of communicating ideas quickly and clearly.
\item Don't share your implementation, you only make it easier for other people
to scoop your ideas.
\item People are interested in the science, not the experimental setup, so don't
describe it.
\item You're \emph{the} expert in the domain, only you can define what
algorithms and data to run experiments with.
\item Any limitations of the method and approximation will be obvious to the
careful reader, no need to waste space on making them explicit.
\end{enumerate}

% Other questions that came up

Are the trustable minimal information for recomputability  different or overlapping for different domains?
\begin{itemize}
\item in AI alone very different
\item require publishing of algorithm and implementation
\item what stepping of the memory controller in systems domain
\end{itemize}

Is there a Dublin Core - or a set of principles - that we can agree for recomputability?

What about a paper which is essentially theoretical but complex, and an implementation is included to show that it is possible to implement it. Should that implementation be reproducible? Arguably not.



%\bibliographystyle{unsrt}
%\bibliography{ITT}

\end{document}
